\usepackage{bcprules}
\usepackage{amsmath}
\usepackage{amsfonts}
\usepackage{amssymb}
\usepackage{amsthm}
\usepackage[mathscr]{eucal}
\usepackage{stmaryrd}
\usepackage{xspace}
\usepackage{mathrsfs}
\usepackage{txfonts}
\usepackage{array}
\usepackage{xargs}
\usepackage[figuresleft]{rotating}
\usepackage{color}
\usepackage{mdframed}
\usepackage{verbatim}

%% misc
\newcommand{\alt}{\ \mid\ }
\newcommand{\lalt}{\ \ \ \alt}
\newcommand{\anywhere}[1]{\ensuremath{\mbox{#1}}}
\newcommand{\kw}[1]{\anywhere{\sffamily{\bfseries\small {#1}}}}
\newcommand{\red}[1]{{\color{red}{#1}}}
\newcommand{\blue}[1]{{\color{blue}{#1}}}
\newcommand{\green}[1]{{\color{green}{#1}}}
\newcommand{\note}[1]{\red{\textbf{#1}}}
\newcommand{\ignore}[1]{}
\newcommand{\ulist}[1]{\begin{itemize} #1 \end{itemize}}
\newcommand{\vsp}{\vspace{.1in}}
\newcommand{\nvsp}{\vspace{-.1in}}
\newcommandx{\framed}[2][1=\nvsp\nvsp]{\begin{mdframed}#1 #2 \nvsp\nvsp\end{mdframed}}
%\newcommandx{\framed}[2][1=\nvsp\nvsp]{#2}
\newcommand{\say}{\vsp\vsp\noindent}

%% fonts
\newcommand{\mtt}[1]{\ensuremath{\mathit{#1}}}
\newcommand{\mrm}[1]{\ensuremath{\mathrm{#1}}}
\newcommand{\ttt}[1]{\ensuremath{\mbox{\small \texttt{#1}}}}

%% syntactic domains
\newcommand{\lang}{\ensuremath{\mathtt{\lambda}_{\mathtt{imp}}}\xspace}
\newcommand{\Int}{\ensuremath{\mathbb{Z}}\xspace}
\newcommand{\Num}{\ensuremath{\mtt{Num}}\xspace}
\newcommand{\Label}{\ensuremath{\mtt{Label}}\xspace}
\newcommand{\Bool}{\ensuremath{\mtt{Bool}}\xspace}
\newcommand{\String}{\ensuremath{\mtt{String}}\xspace}
\newcommand{\Variable}{\ensuremath{\mtt{Variable}}\xspace}
\newcommand{\UnOp}{\ensuremath{\mtt{UnOp}}\xspace}
\newcommand{\BinOp}{\ensuremath{\mtt{BinOp}}\xspace}
\newcommand{\Exp}{\ensuremath{\mtt{Exp}}\xspace}
\newcommand{\Cmd}{\ensuremath{\mtt{Cmd}}\xspace}
\newcommand{\Prog}{\ensuremath{\mtt{Program}}\xspace}
\newcommand{\IType}{\ensuremath{\mtt{InputType}}\xspace}
\newcommand{\Term}{\ensuremath{\mtt{Term}}\xspace}
\newcommand{\Obj}{\ensuremath{\mtt{Object}}\xspace}

%% binary operators
\newcommand{\bop}{\ensuremath{\oplus}}
\newcommand{\logand}{\ensuremath{\land}}
\newcommand{\logor}{\ensuremath{\lor}}
\newcommand{\shl}{\ensuremath{<<}}
\newcommand{\shr}{\ensuremath{>>}}
\newcommand{\shru}{\ensuremath{>>>}}
\newcommand{\binxor}{\ensuremath{\kw{xor}}\xspace}
\newcommand{\binand}{\ensuremath{\kw{and}}\xspace}
\newcommand{\binor}{\ensuremath{\kw{or}}\xspace}
\newcommand{\charAtKW}{\ensuremath{\kw{charAt}}\xspace}

%% unary operators
\newcommand{\uop}{\ensuremath{\odot}}
\newcommand{\floor}{\ensuremath{\kw{floor}}\xspace}
\newcommand{\lognot}{\ensuremath{\kw{not}}\xspace}
\newcommand{\binnot}{\ensuremath{\neg}}
\newcommand{\lengthKW}{\ensuremath{\kw{length}}\xspace}
\newcommand{\keysKW}{\ensuremath{\kw{keys}}\xspace}
\newcommand{\typeofKW}{\ensuremath{\kw{typeof}}\xspace}
\newcommand{\numToStringKW}{\ensuremath{\kw{numToString}}\xspace}
\newcommand{\stringToNumKW}{\ensuremath{\kw{stringToNum}}\xspace}
\newcommand{\hasOwnPropertyKW}{\ensuremath{\kw{hasOwnProperty}}\xspace}

%% syntactic objects
\newcommand{\set}[1]{\ensuremath{\overrightarrow{#1}}}
\newcommandx{\seq}[1][1=e]{\ensuremath{\vec{#1}}}
\newcommandx{\while}[2][1=e_1, 2=e_2]{\ensuremath{\kw{while }#1\; #2}}
\newcommandx{\trycatch}[4][1=e_1, 2=x, 3=e_2, 4=e_3]{\ensuremath{\kw{try }#1 \kw{ catch }#2 \; #3 \kw{ finally }#4}}
\newcommandx{\brk}[2][1=l, 2=e]{\ensuremath{\kw{brk }#1 \; #2}}
\newcommandx{\lbl}[2][1=l, 2=e]{#1 \; #2}
\newcommandx{\throw}[1][1=e]{\ensuremath{\kw{throw }#1}}
\newcommandx{\funcallSyn}[2][1=e_1, 2=\vec{e_2}]{#1(#2)}
\newcommandx{\funcallSynCode}[3][1=e_1, 2=e_2, 2=\vec{e_3}]{#1.#2(#3)}
\newcommandx{\methcallSyn}[2][1=e_1, 2=\vec{e_2}]{#1(#2)}
\newcommandx{\evalSyn}[1][1=e]{\ensuremath{\kw{eval }#1}}
\newcommand{\object}{\ensuremath{\ttt{Object}}\xspace}
\newcommand{\window}{\ensuremath{\ttt{window}}\xspace}
\newcommandx{\cond}[3][1=e_1, 2=e_2, 3=e_3, usedefault]{\ensuremath{\kw{if }#1\; #2\kw{ else } #3}}
\newcommandx{\fun}[3][1=f, 2=\vec{x}, 3=e, usedefault]{\ensuremath{#1(#2) \to #3}}
\newcommand{\str}{\ensuremath{\mtt{str}}\xspace}
\newcommand{\unit}{\kw{unit}\xspace}
\newcommand{\undefinedv}{\kw{undef}\xspace}
\newcommand{\nullv}{\kw{null}\xspace}
\newcommand{\true}{\kw{true}\xspace}
\newcommand{\false}{\kw{false}\xspace}
\newcommandx{\inpt}[1][1=\typ]{\ensuremath{\kw{input }#1}}
\newcommand{\out}[1][e]{\ensuremath{\kw{output }#1}}
\newcommand{\dvar}[1][\vec{x}]{\ensuremath{\kw{var }#1}}
\newcommandx{\block}[2][1=\dvar, 2=e]{\ensuremath{\dvar\kw{ in }#2}}
\newcommand{\typ}{\ensuremath{\mtt{typ}}}
\newcommand{\obj}[1]{\ensuremath{\left\{#1\right\}}}
\newcommand{\method}[1][e]{\ensuremath{(\self, \vec{x}) \to #1}}
\newcommand{\self}{\ttt{self}\xspace}
\newcommandx{\del}[1][1=e_1.e_2]{\kw{del }#1\xspace}

%% semantic domains
\newcommand{\Env}{\ensuremath{\mtt{Env}}\xspace}
\newcommand{\Store}{\ensuremath{\mtt{Store}}\xspace}
\newcommand{\BaseValue}{\ensuremath{\mtt{BaseValue}}\xspace}
\newcommand{\JumpValue}{\ensuremath{\mtt{JumpValue}}\xspace}
\newcommand{\FEIValue}{\ensuremath{\mtt{FEIValue}}\xspace}
\newcommand{\FEOValue}{\ensuremath{\mtt{FEOValue}}\xspace}
\newcommand{\Value}{\ensuremath{\mtt{Value}}\xspace}
\newcommand{\Addr}{\ensuremath{\mtt{Address}}\xspace}
\newcommand{\Conf}{\ensuremath{\mtt{Config}}\xspace}
\newcommand{\Closure}{\ensuremath{\mtt{Closure}}\xspace}
\newcommand{\List}{\ensuremath{\mtt{List}}\xspace}

%% semantic objects
\newcommand{\conf}{\ensuremath{\mathcal{C}}}
\newcommand{\env}{\ensuremath{\rho}}
\newcommand{\EA}{\ensuremath{\Gamma}\xspace}
\newcommand{\store}{\ensuremath{\sigma}}
\newcommand{\eval}{\ensuremath{\Rightarrow}}
\newcommandx{\tup}[3][1=e, 2=\env, 3=\store, usedefault]{\ensuremath{\llbracket #1 \rrbracket #2 #3}}
\newcommandx{\res}[2][1=v, 2=\store_1, usedefault]{\pair{#1}{#2}}
\newcommand{\dom}{\ensuremath{\mtt{dom}}\xspace}
\newcommand{\init}{\ensuremath{\ttt{inject}}\xspace}
\newcommand{\clo}{\ensuremath{\mtt{clo}}\xspace}
\newcommandx{\clot}[2][1=\env, 2=\fun, usedefault]{\ensuremath{#1 \cdot #2}}
\newcommand{\alloc}{\ensuremath{\ttt{alloc}}\xspace}
\newcommand{\allocObj}{\ensuremath{\ttt{allocObj}}\xspace}
\newcommand{\proto}{\ensuremath{\ttt{lookUp}}\xspace}
\newcommand{\firstException}{\ensuremath{\ttt{firstException}}\xspace}
\newcommand{\catchExc}{\ensuremath{\ttt{evalExps}}\xspace}
\newcommand{\catchExcRec}{\ensuremath{\ttt{evalExpsR}}\xspace}
\newcommand{\append}{\ensuremath{\ttt{append}}\xspace}
\newcommand{\repeatFun}{\ensuremath{\ttt{repeat}}\xspace}
\newcommand{\params}{\ensuremath{\ttt{params}}\xspace}
\newcommand{\take}{\ensuremath{\ttt{take}}\xspace}
\newcommand{\length}{\ensuremath{\ttt{length}}\xspace}
\newcommand{\charAt}{\ensuremath{\ttt{charAt}}\xspace}
\newcommand{\typeOf}{\ensuremath{\ttt{typeOf}}\xspace}
\newcommand{\numToString}{\ensuremath{\ttt{numToString}}\xspace}
\newcommand{\stringToNum}{\ensuremath{\ttt{stringToNum}}\xspace}
\newcommand{\hasOwnProperty}{\ensuremath{\ttt{hasOwnProperty}}\xspace}

\newcommandx{\ceRetvalTwo}[3][1=v_1, 2=v_2, 3=\store_{ce}]{\pair{\listtwo[#1][#2]}{#3}}
\newcommandx{\ceRetvalThree}[4][1=v_1, 2=v_2, 3=v_3, 4=\store_{ce}]{
   \pair{\listthree[#1][#2][#3]}{#4}}
\newcommand{\first}{\ensuremath{\ttt{first}}\xspace}
\newcommand{\rest}{\ensuremath{\ttt{rest}}\xspace}
\newcommand{\last}{\ensuremath{\ttt{last}}\xspace}
\newcommand{\head}{\ensuremath{\ttt{head}}\xspace}
\newcommand{\lastExc}[2]{\assign{\pair{j}{#2}}{#1}}
\newcommand{\updateObj}{\ensuremath{\ttt{updateObj}}\xspace}
\newcommand{\updateObjAddr}{\ensuremath{\ttt{updateObjAddr}}\xspace}
\newcommand{\foldLeft}{\ensuremath{\ttt{foldLeft}}\xspace}
\newcommand{\objKeys}{\ensuremath{\ttt{objKeys}}\xspace}
\newcommand{\objKeysAll}{\ensuremath{\ttt{objKeysAll}}\xspace}
\newcommand{\recKeys}{\ensuremath{\ttt{recKeys}}\xspace}
\newcommand{\tryJmpValue}{\ensuremath{\ttt{tryJmpValue}}\xspace}
\newcommand{\tryExcValue}{\ensuremath{\ttt{tryExcValue}}\xspace}
\newcommand{\reverse}{\ensuremath{\ttt{reverse}}\xspace}
\newcommand{\call}{\ensuremath{\ttt{call}}\xspace}
\newcommand{\accOthValue}{\ensuremath{\ttt{accOthValue}}\xspace}
\newcommandx{\exc}[1][1=v]{\kw{exc }#1}
\newcommandx{\jmp}[2][1=l, 2=v]{\kw{jmp }#1\;#2}
\newcommand{\nil}{\kw{nil}}

\newcommand{\protoExternal}{``\ttt{prototype}"\xspace}
\newcommand{\protoInternal}{``\ttt{\_\_proto\_\_}"\xspace}
\newcommand{\key}{``\ttt{key}"\xspace}
\newcommand{\nextRec}{``\ttt{next}"\xspace}

\newcommand{\truthy}{\ensuremath{\ttt{truthy}}\xspace}
\newcommand{\falsy}{\ensuremath{\ttt{falsy}}\xspace}
\newcommand{\parse}{\ensuremath{\ttt{parse}}\xspace}

% misc
\newcommandx{\TEExc}[1][1=\store_1]{\res[{\exc[\typeError]}][#1]}
\newcommandx{\listtwo}[2][1=e_1, 2=e_2]{#1\cdot#2}
\newcommandx{\listthree}[3][1=e_1, 2=e_2, 3=e_3]{#1\cdot#2\cdot#3}

\newcommand{\pair}[2]{\ensuremath{(#1, #2)}}
\newcommand{\feinput}{\set{\pair{g}{\store}}}
\newcommand{\ruleBlock}{\vsp\vsp\vsp}
\newcommand{\subscr}[1]{\textsubscript{#1}}
\newcommandx{\nth}[1][1=n]{\ensuremath{#1}th\xspace}
%\newcommand{\assign}[2]{#1 \triangleq #2}
\newcommand{\assign}[2]{#1 = #2}
\newcommand{\assignRev}[2]{\assign{#2}{#1}}
\newcommand{\typeError}{``\mtt{TypeError}"}
\newcommand{\notImpl}{``\mtt{NotImplemented}"}
\newcommand{\todo}{\red{TODO:}\xspace}
\newcommandx{\objectPrototype}[2][1=\env, 2=\store]{#2(#2(#2(#1(\object)))( \protoExternal ))}
\newcommand{\nullUndef}{\{ \nullv, \undefinedv \}}

\newcommand{\code}[1]{\ensuremath{\mathtt{#1}}\xspace}
\newcommand{\av}[2]{\ensuremath{\{v_{#1}\ |\ \psi_{#2}(v_{#1})\}}\xspace}
\newcommand{\pr}[2]{\ensuremath{\psi_{#2}(v_{#1})}\xspace}
