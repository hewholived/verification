\subsection{Abstract semantics}
\begin{description}


\item[Let block rule:]
Initialize the abstract values of all the variables introduced in the let block to the least precise value \ensuremath{\top}, i.e. \ensuremath{\{v\ |\ \true\}} and update the store with these mappings. Update \marketplace with the new set of logical qualifiers corresponding to each of the newly introduced variables.


\item[Constant rule:]
Any constant integer \textit{i} has an abstract value \ensuremath{\{v\ |\ v=i\}}.


\item[Variable rule:]
Lookup most recent abstract value of the variable from the store.


\item[Assign rule:]
For any assignment of the form \ensuremath{x := e}\ first evaluate \textit{e} to an abstract value \av{}{}. Then refine the abstract value of variable \textit{x} using the following algorithm.
\begin{enumerate}
\item
Assume \ensuremath{\alpha} equals $(x_{new} = v) \wedge \pr{}{} \wedge \EA$, where $x_{new}$ is a unique variable name associated with \textit{x} at the given program location and \EA is the assumption that is made in the environment.
\item
For each logical qualifier \ensuremath{\beta} in \marketplace, check whether \ensuremath{\alpha \implies \beta[v\mapsto x_{new}]} is valid and filter out the ones that invalidate the formula into a set $\Sigma$. Filter out a set of logical qualifiers $\Lambda$ which contain the relationship between old values of $x$, $x_{old}$ with $x_{new}$.
\item
Create a new abstract value \av{1}{1} for \textit{x}, where \pr{1}{1} is the conjunction of all the logical qualifiers in $\Sigma - \Lambda$. Update the store with this new value.
\item
Update the values of variables in the store which depended on $x_{old}$ using the relations in $\Lambda$.
\item
Check the validity of \EA using the constraints in $\Sigma$. If \EA is invalid, replace it with \true.
\end{enumerate}


\item[Binary operator rule:]
For any binary operation of the form \ensuremath{e_1 \bop e_2}, evaluate $e_1$ and $e_2$ to \av{1}{1} and \av{2}{2}. Create a new abstract value \av{3}{3} using the following algorithm.
\begin{enumerate}
\item
Assume $\alpha$ equals $\pr{1}{1} \logand \pr{2}{2} \logand (v_3 = v_1 \bop v_2)$.
\item
For each logical qualifier \ensuremath{\beta} in \marketplace, check whether \ensuremath{\alpha \implies \beta[v\mapsto v_3]} is valid and filter out the ones that invalidate the formula into a set $\Sigma$.
\item
Create a new abstract value \av{3}{3}, where $v_3$ is fresh variable name, \pr{3}{3} is the conjunction of all the logical qualifiers in $\Sigma$.
\end{enumerate}


\item[If rule:] 
For an \kw{if} expression of the format \cond , use the following algorithm to evaluate it. We assume that $e_1$ is of the form $\textit{lexp} \odot \textit{rexp}$.
\begin{enumerate}
\item
Evaluate \textit{lexp} and \textit{rexp} to abstract values \av{1}{1} and \av{2}{2}.
\item
Let\\ 
$\textit{cond} := \pr{1}{1} \logand \pr{2}{2} \logand \EA$ \\
$\textit{isTrue} := \kw{sat?}((v_1 \odot v_2) \logand cond)$ \\
$\textit{isTrue} := \kw{sat?}(!(v_1 \odot v_2) \logand cond)$ \\
\\
where \kw{sat?} is a function that tests the satisfiability of a given expression.
\item
Obtain new stores $\store_t$ and $\store_f$ by evaluating $e_2$ and $e_3$ as follows.\\
$\store_t := \kw{eval}(e_2, \store, \EA\ \kw{extend}\ ((v_1 \odot v_2) \logand cond))$\\
$\store_f := \kw{eval}(e_3, \store, \EA\ \kw{extend}\ (!(v_1 \odot v_2) \logand cond))$
\item
Obtain $\store_1$ and $\store_2$ using the formula \\
$\store_1 := \textit{isTrue}\ ?\ \store_t\ :\ \store_f$\\
$\store_2 := \textit{isFalse}\ ?\ \store_f\ :\ \store_t$
\item
Obtain the final store by merging $\store_1$ and $\store_2$\\
$\store_{final} := \store_1 \bigsqcup\ \store_2 $
\end{enumerate}


\item[While rule:] 
For a \kw{while} expression of the format \while , use the following algorithm to evaluate it. We assume that $e_1$ is of the form $\textit{lexp} \odot \textit{rexp}$. Let $\store_s = \store_1 = \store_2$ be the current store. 
\begin{enumerate}
\item
$\store_s = \store_1 \bigsqcup \store_2$
\item
Evaluate \textit{lexp} and \textit{rexp} to abstract values \av{1}{1} and \av{2}{2}.
\item
Let\\ 
$\textit{cond} := \pr{1}{1} \logand \pr{2}{2} \logand \EA$ \\
$\textit{isTrue} := \kw{sat?}((v_1 \odot v_2) \logand cond)$ \\
$\textit{isTrue} := \kw{sat?}(!(v_1 \odot v_2) \logand cond)$ \\
\\
where \kw{sat?} is a function that tests the satisfiability of a given expression.
\item
Obtain new stores $\store_b$ by evaluating $e_2$ as follows.\\
$\store_b := \kw{eval}(e_2, \store, \EA\ \kw{extend}\ ((v_1 \odot v_2) \logand cond))$
\item
Obtain $\store_1$ and $\store_2$ using the formulae \\
$\store_1 := \textit{isTrue}\ ?\ \store_b\ :\ \store_s$\\
$\store_2 := \textit{isFalse}\ ?\ \store_s\ :\ \store_b$
\item
If $(\store_1 \bigsqcup \store_2) \neq \store_s$; goto 1. Else we have reached a fixpoint.
\item
Re-evaluate $e_1$ in $\store_s$ and get the predicate $\textit{fPred} = !(v_1 \odot v_2) \logand cond$. Extend \EA with \textit{fPred}.
\end{enumerate}
\end{description}

\section{A Simple Theory for Multiplication}
Modern day SMT solvers do not handle non-linear arithmetic on variables well. To solve this problem, we decided to encode a theory for multiplication in our toolkit. Since the set of logical qualifiers we deal with is limited, we were able to come up with a fixed set of rules that produce a sound abstract value of a product. We calculate the product of two abstract values \av{1}{1} and \av{2}{2} using the following rules.\\\\
$\textit{case}\ (v_1 = 0 \logor v_2 = 0):\ \{v_3\ |\ v_3 = 0\}$\\
$\textit{case}\ \av{1}{1} = \av{2}{2}:\ \{v_3\ |\ (v_3 > 0) \logand (v_3 \ge v_1)\}$\\
$\textit{case}\ (v_1 > 0 \logand v_2 > 0):\ \{v_3\ |\ (v_3 > 0) \logand (v_3 \ge v_1) \logand (v_3 \ge v_2)\}$\\
$\textit{case}\ (v_1 > 0 \logand v_2 < 0):\ \{v_3\ |\ (v_3 < 0) \logand (v_3 < v_1) \logand (v_3 \le v_2)\}$\\
$\textit{case}\ (v_1 < 0 \logand v_2 > 0):\ \{v_3\ |\ (v_3 < 0) \logand (v_3 \le v_1) \logand (v_3 < v_2)\}$\\
$\textit{case}\ (v_1 > 0 \logand v_2 \ge 0):\ \{v_3\ |\ (v_3 \ge 0) \logand (v_3 \ge v_2)\}$\\
$\textit{case}\ (v_1 > 0 \logand v_2 \le 0):\ \{v_3\ |\ (v_3 \le 0) \logand (v_3 < v_1) \logand (v_3 \le v_2)\}$\\
$\textit{case}\ (v_1 < 0 \logand v_2 \ge 0):\ \{v_3\ |\ (v_3 \le 0) \logand (v_3 \le v_2)\}$\\
$\textit{case}\ (v_1 < 0 \logand v_2 \le 0):\ \{v_3\ |\ (v_3 \ge 0) \logand (v_3 > v_1) \logand (v_3 \ge v_2)\}$\\
$\textit{case}\ (v_1 \ge 0 \logand v_2 > 0):\ \{v_3\ |\ (v_3 \ge 0) \logand (v_3 \ge v_1)\}$\\
$\textit{case}\ (v_1 \ge 0 \logand v_2 < 0):\ \{v_3\ |\ (v_3 \le 0) \logand (v_3 \le v_1)\}$\\
$\textit{case}\ (v_1 \le 0 \logand v_2 > 0):\ \{v_3\ |\ (v_3 \le 0) \logand (v_3 \le v_1) \logand (v_3 < v_2)\}$\\
$\textit{case}\ (v_1 \le 0 \logand v_2 < 0):\ \{v_3\ |\ (v_3 \ge 0) \logand (v_3 \ge v_1) \logand (v_3 > v_2)\}$\\
$\textit{case}\ (v_1 \ge 0 \logand v_2 \ge 0):\ \{v_3\ |\ (v_3 \ge 0) \logand (v_3 \ge v_1) \logand (v_3 \ge v_2)\}$\\
$\textit{case}\ (v_1 \ge 0 \logand v_2 \le 0):\ \{v_3\ |\ (v_3 \le 0) \logand (v_3 \le v_1)\}$\\
$\textit{case}\ (v_1 \le 0 \logand v_2 \ge 0):\ \{v_3\ |\ (v_3 \le 0) \logand (v_3 \le v_2)\}$\\
$\textit{case}\ (v_1 \le 0 \logand v_2 \le 0):\ \{v_3\ |\ (v_3 \ge 0) \logand (v_3 \ge v_1) \logand (v_3 \ge v_2)\}$\\
$\textit{default} : \{v_3\ |\ \true\}$