\section{Logical Qualifier Inference through Abstract Interpretation}

We infer flow-sensitive logical qualifiers for each variable at each program point by using abstract interpretation.
In this section, we describe the lattice for concrete collecting semantics, and the lattice for abstract semantics, provide a galois connection between the two, and then provide rules for abstract semantics informally.

\subsection{Abstract domain}
The lattice for concrete collecting semantics (which provides all possible values an integer variable in \lang could be, at a given program point for a given program) is given by 
\begin{description}
\item[Element] A set of integers
\item[$\top$] Set of all integers
\item[$\bot$] Empty set
\item[Partial Ordering] Subset inclusion
\item[Join] Set union
\item[Meet] Set intersection
\item[Equality] Set equality
\end{description}

The lattice for abstract semantics (the elements of which represent the abstract values over which the abstract semantics computes) is given by:
\begin{description}
\item[Element] \av{}{} where \pr{}{} is a logical qualifier for $v$
\item[$\top$] $\{ v\ |\ \textit{true}\}$
\item[$\bot$] $\{ v\ |\ \textit{false}\}$
\item[Partial Ordering] \av{}{1} $\sqsubseteq$ \av{}{2} iff \pr{}{1} $\rightarrow$ \pr{}{2}
\item[Join] $\av{}{1} \sqcup \av{}{2} = \av{}{3}$ iff $\pr{}{1} \rightarrow \pr{}{3}\ \land\ \pr{}{2} \rightarrow \pr{}{3}$ and there is no $\pr{}{4} \neq \pr{}{3}$ such that $\pr{}{1} \rightarrow \pr{}{4}\ \land\ \pr{}{2} \rightarrow \pr{}{4} \ \land \ \pr{}{4} \rightarrow \pr{}{3}$
\item[Meet] $\av{}{1} \sqcap \av{}{2} = \av{}{3}$ iff $\pr{}{3} \rightarrow \pr{}{1}\ \land\ \pr{}{3} \rightarrow \pr{}{2}$ and there is no $\pr{}{4} \neq \pr{}{3}$ such that $\pr{}{4} \rightarrow \pr{}{1}\ \land\ \pr{}{4} \rightarrow \pr{}{2} \ \land \ \pr{}{3} \rightarrow \pr{}{4}$
\item[Equality] $\av{}{1} = \av{}{2}$ iff $\pr{}{1} \leftrightarrow \pr{}{2}$
\end{description}

Observe that because the abstract values are logical qualifiers (which cannot have disjunctions), when $\av{}{1} = \av{}{2}$, they must be syntactically equivalent too.

\newtheorem{theorem}{Theorem}
\begin{theorem}
\label{theo:join}
\emph{(Unique Join)} The join of any two elements $\av{}{1}$ and $\av{}{2}$ is unique. 
\end{theorem}
\begin{proof}
Let us assume that there are two distinct joins for elements $\av{}{1}$ and $\av{}{2}$.
Let they be $\av{}{3}$ and $\av{}{4}$.
By applying the definition of $\av{}{1} \sqcup \av{}{2}$, to each of these, and using the fact that they are distinct, we get, $\pr{}{1} \rightarrow \pr{}{3}\ \land\ \pr{}{2} \rightarrow \pr{}{3}\ \land\ \pr{}{4} \neq \pr{}{3}\ \land\ \pr{}{3} \centernot\rightarrow \pr{}{4}\ \land\ \pr{}{1} \rightarrow \pr{}{4}\ \land\ \pr{}{2} \rightarrow \pr{}{4}\ \land\ \pr{}{4} \centernot\rightarrow \pr{}{3}$.
But this is unsatisfiable (can be verified by writing the truth table or feeding to a theorem solver). 
Thus, the assumption we made, that there are two distinct joins must be incorrect.
Therefore, the join of any two elements $\av{}{1}$ and $\av{}{2}$ is unique.
\end{proof}

Using Theorem~\ref{theo:join}, we can confirm that definition of the abstraction lattice we provide is indeed a lattice.

\begin{theorem}
\label{theo:galois}
\emph{(Galois connection)}
There exists a galois connection between the concrete lattice and abstract lattice defined above.
\end{theorem}
\begin{proof}
Let the abstraction function $\alpha$ be given by $\alpha(I) = \av{}{1}$ where $I \subseteq \mathbb{Z}$ and $\forall i \in I, \psi_1(i)$ is true and there is no $\av{}{2}$ such that $\pr{}{2} \rightarrow \pr{}{1}$ and $\forall i \in I, \psi_2(i)$ is true.
Let the concretization function $\gamma$ be given by $\gamma(\av{}{})$ = $\{ i \ |\ i \in \mathbb{Z} \land \psi(i) \}$.
Using these, we can see that there exists a galois connection between the concrete and the abstract lattice.
\end{proof}
